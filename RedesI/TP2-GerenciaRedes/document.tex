\documentclass[brazil,times,12pt]{abnt}
\usepackage[T1]{fontenc}
\usepackage[utf8]{inputenc}
\usepackage{url}
\usepackage{graphicx}
\usepackage[pdfborder={0 0 0}]{hyperref}
\makeatletter
\usepackage{babel}
\makeatother
\begin{document}

\autor{Pedro Paulo Vezzá Campos}

\titulo{Prática de Gerência de Redes Usando Ferramenta SNMP}

\comentario{Trabalho apresentado para avaliação na disciplina INE5414, do
curso de Bacharelado em Ciências da Computação, turma 04208, da Universidade   
Federal de Santa Catarina, ministrada pelo professor Carlos Becker Westphall}

\instituicao{Departamento de Informática e Estatística \par Centro
Tecnológico \par Universidade Federal de Santa Catarina}

\local{Santa Catarina - SC, Brasil}

\data{\today}

\capa

\folhaderosto

% \tableofcontents
%\chapter{}
\section*{Introdução à Gerência de Redes}
A gerência de redes está associada ao controle de atividades e ao monitoramento
do uso de recursos da rede. Suas atribuições incluem: Obter informações da
rede; Tratar estas informações, possibilitando um diagnóstico; E encaminhar as
soluções dos problemas. Para cumprir estes objetivos, funções de gerência devem
ser embutidas nos diversos componentes de uma rede, possibilitando descobrir,
prever e reagir a problemas. \cite{duarte:gerencia_redes}

\subsection*{Modelos de gerência}
Há vários modelos diferentes de gerência de redes. Na gerência centralizada, um
único processo gerente controla a gerência. Pela sua natureza centralizada, esse
modelo é vulnerável em redes grandes. Já na gerência descentralizada as
atividades de gerência são distribuídas, e vários nós da rede rodam gerentes. O
gerenciamento também pode ser feito de forma hierárquica, nesse sentido, cada
nó fica sendo responsável por um tipo de atividade gerencial. \cite{wiki:network_management}

Na gerência reativa há o alerta aos administradores dos problemas ocorridos,
passando a atuar em sua solução. Por outro lado, na gerência pró-ativa os
processos gerentes também alteram as configurações dos dispositivos e recursos
da rede como forma de auto-gerência.

\subsection*{Termos da Gerência de Redes}
\begin{description}
\item[Gerente] Um computador conectado a rede que executa o software de protocolo de gerenciamento
que solicita informações dos agentes. O sistema de gerenciamento também é chamado de console de 
gerenciamento.
\item[Agente] Um processo (software) que roda em um recurso, elemento ou sistema gerenciado, que exporta
uma base de dados de gerenciamento (MIB) para os que o gerente possa ter acesso aos mesmos.
\item[MIB] Management Information Base – Base de dados de gerenciamento – é uma tabela onde são 
armazenados os dados de gerenciamento coletados que serão enviados ao gerente.
\item[Protocolo de gerenciamento] Fornece os mecanismos de comunicação entre o
gerente e o agente. \cite{wiki:gerencia_redes}
\end{description}
	
\section*{O protocolo SNMP}
O protocolo SNMP (\emph{Simple Network Management Protocol}) faz parte do
conjunto de protocolos da Internet, como foi definido pela \emph{Internet
Engineering Task Force} ou IETF. Ele é usado em sistemas de gerenciamento para
monitorar dispositivos de rede procurando por condições que possuam necessidade
de administração. O SNMP consiste de uma série de padrões, incluindo um
protocolo de nível de aplicação e um sistema de banco de dados. Seu
funcionamento deriva do modelo de hierarquisas da orientação a objetos.

No protocolo SNMP, as informações sobre a rede são expostas na forma de
variáveis nos sistemas gerenciados, que descrevem a configuração desses mesmos
sistemas. Essas variáveis podem ser consultadas e (possivelmente) alteradas
pelos processos gerentes. \cite{wiki:snmp}

\section*{Experimento Prático}
Para este trabalho prático foi realizado o acompanhamento de variáveis SNMP em
um \emph{access-point} Linksys WRT54G rodando o firmware dd-wrt em sua última
versão estável. Suas funções são de fornecer acesso à Internet via Internet sem
fio (802.11g), além de cumprir tarefa como servidor DHCP.

O próprio firmware já possuia embutido um \emph{daemon} SNMP embutido, sendo
somente necessário habilitá-lo via interface web e configurá-lo via SSH,
conforme instruções em: \cite{ddwrt:snmp}. Como ferramenta para visualizar os
dados informados pelo \emph{access-point} foi utilizada a ferramenta Getif.

O monitoramento do servidor foi realizado durante aproximadamente 24 horas e os
valores coletados foram os seguintes:

\begin{center}
% use packages: array
\begin{tabular}{ll}
Variável & Valor \\ 
tcpOutRsts:  & 16811 \\
tcpAttemptFails:  & 1774 \\ 
tcpInSegs:  & 185707102 \\ 
tcpOutSegs:  & 87452074 \\
tcpActiveOpens:  & 7538 \\  
tcpRetransSegs:  & 98771 
\end{tabular}
\end{center}

\bibliographystyle{abnt-num}
\bibliography{bibliografia}
\end{document}