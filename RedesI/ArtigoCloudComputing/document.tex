\documentclass[brazil,12pt]{article}
\usepackage{times}
\usepackage[T1]{fontenc}
\usepackage[utf8]{inputenc}
\makeatletter
\usepackage{babel}
\makeatother
\usepackage[pdfborder={0 0 0}]{hyperref}

\begin{document}

\title{Um Panorama das Técnicas de Segurança em Cloud Computing}
\author{Pedro Paulo Vezzá Campos}
\date{\today}
\maketitle

\begin{abstract}
A Computação em Nuvem, ou cloud computing vem avançando tanto no mercado
empresarial quanto para usuários finais como uma forma de
reduzir custos e facilitar a manutenção ao delegá-los a empresas especializadas.
Porém, ao mesmo tempo surgem diversos questionamentos quanto à segurança dos
dados confiados a terceiros. Neste artigo em formato de \emph{survey} será
abordado o estado da arte nas técnicas de proteção produzidas para cloud computing.
\end{abstract}

\section{Motivação}
Cloud computing é um novo nicho de mercado que tornou-se mais atrativo com o
barateamento de insumos necessários à computação como energia, poder de
processamento, armazenamento e transmissão de dados, permitindo uma economia de
escala. \cite{above-clouds} Através da computação em nuvem, surgiu a noção de
computação como um serviço, elástico, virtualmente ilimitado e pago apenas pela
porção realmente utilizada, muito similar ao sistema de distribuição elétrica.

Diante desse cenário, \emph{players} como Amazon, Google,
Microsoft, HP, IBM, dentre outros, adentraram essa área oferecendo diversas
modalidades de cloud computing, desde a mais abstrata, SaaS ou Software como
Serviço, na qual é fornecido um software pronto para ser utilizado utilizando
recursos da nuvem, até a mais básica, IaaS ou Infraestrutura como Serviço, que
permitem uma personalização da computação desde o nível de \emph{kernel} até
camadas superiores.

Por outro lado, críticas e dúvidas são levantadas por opositores dessa
tecnologia. Um exemplo é Richard Stallman, evangelista do software livre, que
afirma em uma tradução livre:

\begin{quote}
``É estupidez. É pior que estupidez: É uma campanha de marketing. Alguém está
dizendo que isso é inevitável — e sempre que você ouvir alguém falando isso, é
muito provável que seja um conjunto de empresas realizando campanha para
torná-lo verdade.'' \cite{stallman-cloud}
\end{quote}

Paralelamente às oportunidades possibliltadas através do uso de cloud computing
há um aumento das preocupações com a segurança dos dados armazenados na núvem.
Frente a isso diversas pesquisas abordaram técnicas tradicionais e inovadoras
para solucionar esse problema. Neste trabalho serão apresentadas as mais
importantes dessas propostas apresentadas.

Este artigo está organizado da seguinte maneira: A seção 2 deste artigo trata
dos objetivos visados na elaboração desse texto. Na parte 3 há um estudo dos
principais trabalhos correlatos na área de cloud computing. Já na seção 4 são
apresentadosa os principais conceitos relacionados à area, necessários para a
compreenção dos assuntos seguintes. Posteriormente, na seção 5 são apresentadas
as principais preocupações de usuários da nuvem relativos à sua segurança.
Continuando, a parte 6 aborda as principais técnicas tradicionais de segurança,
aplicáveis à cloud. Ainda, a parte 7 explora as novas possibilidades de ataques
e defesas originados do uso (e abuso) da computação em nuvem. Por fim, a parte 8
apresenta as conclusões mais importantes desse trabalho.

\section{Objetivos}
O principal objetivo deste \emph{survey} é apresentar um panorama geral das
técnicas de segurança em cloud computing, apresentando tanto ataques possíveis a
um usuário ou gerente da nuvem quanto maneiras de proteger-se de tais invasões.
Para isso, serão apresentados inicialmente os conceitos principais da área
de cloud computing como forma de fundamentação. Posteriormente, serão
apresentados os mais importantes tópicos de preocupação de usuários e
administradores de sistemas a respeito da computação nas nuvens. Por fim, será
apresentado um paralelo entre técnicas de ataque e defesa existentes.

\section{Trabalhos Correlatos}

Em \cite{controlling-data-in-cloud} são apresentados diversas preocupações com
cloud computing, categorizadas segundo três macro áreas, como apresentado
abaixo. Ao longo desse trabalho serão apresentados e discutidos os itens contidos em cada uma dessas
categorias.
\begin{itemize}
  \item Segurança tradicional
  \item Disponibilidade
  \item Controle de dados por terceiros
\end{itemize}

Em \cite{whats-new-about-cloud-security}, Chen et. al. fazem um paralelo entre
os desafios tradicionais enfrentados pelos provedores de cloud computing e novas
possibilidades ataques. O trabalho é permeado por exemplos de técnicas de
proteção em sistemas históricos, como o Multics, comprometimento de sistemas e
suas consequências, dentre outros.

Ristenpart et. al. apresentam em \cite{hey-you-get-off-of-my-cloud} um
exemplo prático de uma tentativa de invasão em um serviço de computação em
nuvem, tomando como exemplo o Amazon EC2. No trabalho os autores apresentam
detalhadamente as técnicas empregadas para mapear a rede que compõe a nuvem da
empresa, heurísticas adotadas para conseguir que uma máquina virtual invasora
seja instalada na mesma máquina física que a máquina virtual alvo e, por fim,
explorando vulnerabilidades da ferramenta de virtualização conseguir gerar um
canal lateral entre máquinas virtuais, burlando o isolamento imposto pelo
\emph{hipervisor}. 

\section{Conceitos básicos}
O significado de cloud computing ainda é motivo de discussão na indústria e
academia, uma vez que seu significado é amplo. Uma das tentativas de
sistematizar a definição desse termo pode ser encontrada em \cite{above-clouds}.
Ainda, segundo o \emph{National Institute of Standards and Technology}, algumas
das características essenciais do cloud computing incluem
\cite{nist-definition-cloud-computing}:

\begin{itemize}
  \item Serviço sob demanda
  \item Acesso em banda larga
  \item \emph{Pooling} de recursos
  \item Elasticidade rápida
  \item Consumo medido, similar ao do sistema de distribuição elétrica, por
  exemplo.
\end{itemize}

\subsection{Modelos de serviços}
Atualmente há três modelos de serviços reconhecidos como cloud computing, cada
um variando no nível de configurabilidade e abstração, como é possível ver
abaixo:

\begin{description}
  \item[SaaS - Software-as-a-Service] SaaS, ou Software como um Serviço, é a
  modalidade de cloud computing mais abstrata de todas. O usuário possui apenas
  a capacidade de configurar a aplicação utilizada. Nesta faixa encontram-se
  serviços tais como o Gmail, Google Apps, Evernote, etc.
  \item[PaaS - Platform-as-a-Service] PaaS, ou Plataforma como um Serviço, é a
  versão intermediária de computação em núvem. Nesta, o usuário possui o poder
  configurar certas varáveis do ambiente de hospedagem. Aqui encontram-se
  serviços tais como o Google App Engine.
  \item[IaaS - Infrastructure-as-a-Service] IaaS, ou Infraestrutura como um
  Serviço, é a versão mais básica de computação em núvem. Neste ponto o usuário
  tem total controle do ambiente desde o kernel até camadas mais superires. Um
  exemplo de IaaS é o Amazon EC2.
\end{description}

Ainda, há variadas maneiras de implementar uma núvem, variando no
compartilhamento de recursos entre diferentes clientes:

\begin{description}
  \item[Cloud Pública] Nesse modelo qualquer pessoa ou empresa pode contratar o
  serviço de cloud computing.
  \item[Cloud Comunitáira] Aqui o serviço é fornecido a apenas algumas
  organizações.
  \item[Cloud Privada] A núvem passa a ser exclusiva de uma organização.
  \item[Cloud Híbrida] Uma mistura dos modelos anteriores.
\end{description}

\section{Principais preocupações}
A natureza da nuvem, uma estrutura normalmente externa ao \emph{firewall} da
organização e possívelmente gerenciada por terceiros, gera a necessidade de
estudos detalhados de análise de riscos antes de adotar uma solução de cloud
computing. Algumas das preocupações apresentadas por
\cite{controlling-data-in-cloud} estão descritas abaixo:

\subsection{Segurança tradicional}
\begin{description}
  \item[Ataque à virtualização] Os hipervisores de virtualização são pontos
  sensíveis a um ataque direcionado, uma vez que eles são responsáveis por
  garantir o isolamento entre máquinas virtuais. Vulnerabilidades importantes já
  foram encontradas em diversas ferramentas, como apresenta 
  \cite{controlling-data-in-cloud}.
  \item[Maior superfície de ataque] Com a cloud encontrando-se possivelmente
  fora do firewall da empresa há a necessidade de haver uma proteção da
  infraestrutura, principalmente a rede que os conectam.
  \item[Vulnerabilidades do provedor de cloud] Outra possibilidade é a invasão
  ser dirigida ao provedor de cloud computing. Um ataque pode comprometer a
  infraestrutura de autenticação do sistema e assim permitir o invasor o acesso
  irrestrito a dados sensíveis do usuário.
\end{description}

\subsection{Disponibilidade}
\begin{description}
  \item[Controle da integridade computacional] Ao contratar um serviço de cloud
  computing o cliente espera que o fornecedor da cloud aja de boa-fé, fornecendo
  valores de computação corretos. Dependendo do nível de criticidade de um
  serviço localizado na nuvem, apenas essa promessa é insuficiente. Para
  resolver esse problema pode-se adotar a redundância, por exemplo. O projeto
  Folding@Home envia tarefas idênticas para computadores diferentes no intuito
  de atingir um consenso nos resultados obtidos. A Google, por outro lado,
  armazena cópias de dados em diversas máquinas diferentes como forma de
  tolerância a falhas.
\end{description}


\subsection{Controle de dados por terceiros}
\begin{description}
  \item[Aprisionamento de dados] Uma das grandes preocupações com o cloud
  computing é que um cliente passe a ser dependente de um fornecedor único (ou
  um pequeno grupo) de forma que o primeiro passe a ser um refém do segundo.
  Esse problema pode surgir através da imposição do uso de formatos ou APIs
  proprietárias. O Google App Engine, por exemplo, impõe o uso de tecnologias
  internas à Google, como o BigTable, GFS, dentre outros. Claramente uma solução
  a esse problema é a adoção de padrões livres. Uma amostra dessa tentativa é a
  API GoGrid.
\end{description}

%   \item[Auditabilidade] 
%   VM-level attack
% Cloud provider vulnerabilities.
% Expanded network attack surface
% Authentication and Authorization
% Auditability.
% Cloud Provider Espionage.

\section{Técnicas tradicionais}
Na área de técnicas tradicionais de segurança, pode-se incluir a criptografia
como forma de controlar o acesso a informações privilegiadas armazenadas na nuvem. Ainda, o
fator humano continua sendo crucial, ataques de engenharia social continuam
sendo bastante efetivos em conseguir comprometer um sistema.

Um exemplo dessa última afirmação foi apresentada na conferência BlackHat USA
2009. Invasores tiveram acesso a máquinas virtuais instaladas no Amazon EC2
apenas fornecendo imagens de sistemas operacionais infectadas de maneira
aparentemente oficial, utilizando nomes como ``fedora\_core''.

Por outro lado, as organizações devem estar preparadas para a recuperação de
disastres, dessa forma, backups são essenciais. A ausência de um cuidado simples
pode levar a sérias consequências como é possível encontrar na bibliografia.

No Brasil um caso recente foi o do migre.me: Uma falha catastrófica no servidor
que mantinha o serviço causou não só a perda de todos o banco de 
redirecionamentos quanto os backups do site. Como agravante, o administrador 
não mantinha backups próprios dos dados do site, apenas do código. A
consequência disso é que milhares de URLs do migre.me tornaram-se
irrecuperáveis, causando diversos transtornos e possivelmente prejuízos aos
usuários do serviço.


\section{As novas possibilidades}
Uma prova da importância da área de segurança voltada para o cloud computing
é a criação de eventos importantes na área tais como: ACM Cloud Computing
Security Workshop e a ACM Conference on Computer and Communications Security.
Nelas são discutidas diferentes abordagens para aprimorar as defesas dos
serviços contra ataques cada vez mais sofisticados.

Nesta seção serão apresentados dois tópicos um mostrando as últimas técnicas de
ataques ao cloud computing e outro apresentando o estado da arte
das técnicas de defesa na área.

\subsection{Novos ataques}
No contexto de cloud computing, uma possibilidade pouco explorada anteriormente
é a de captura do tráfego gerado e a transmissão ativa de dados por um agente
malicioso. Isso é possível graças ao compartilhamento de recursos realizado pela
nuvem. Outra possiblidade é a de um invasor conseguir recuperar informações
sobrescritas ou apagadas por um cliente.

Ainda, com a concentração de vários clientes diferentes em somente um local, um
ataque massivo pode afetar diversos usuários ao mesmo tempo. Um caso
exemplificável foi o \emph{blacklisting} de uma grande faixa de endereços IP do
Amazon EC2 para o envio de emails depois que spammers conseguiram subverter as
proteções levantadas pela empresa para evitar abusos do serviço. Vários usuários
sofreram interrupção na entrega normal de seus emails. Como medida para corrigir
o problema, agora os clientes devem preencher um formulário junto à Amazon
informando a necessidade de envio de grandes quantidades de emails para que seja
requisitado o \emph{whitelisting} da faixa de IPs correspondente àquele usuário
específico.

\subsection{Novas defesas}
Uma crescente vantagem da computação em núvem é a concentração de
expertise em segurança. Um único provedor de cloud computing pode diluir os
custos para manter políticas de segurança e uma equipe de segurança responsável
por controlar possíveis falhas e repará-las.

Isso leva a uma modalidade de serviços conhecida como
\emph{Security-as-a-Service} na qual um provedor fornece soluções prontas de
segurança tais como frameworks, criptografia, antivírus, etc.

\section{Conclusão}
Como foi possível perceber ao longo do artigo a área de segurança em cloud
computing alia tanto conceitos tradicionais de segurança de redes, quanto novas
técnicas de proteção. Ela ainda permanece em constante evolução à medida que
novas técnicas de ataques surgem e novas soluções a essas invasões são
implementadas.

Através de diferentes exemplos práticos envolvendo diversas empresas
fornecedoras de serviços de computação em nuvem foi possível ver as
consequências severas que podem surgir a partir de vulnerabilidades e descuidos
tanto dos administradores da cloud quanto dos usuários dos serviços. Isso
corrobora a importância das pesquisas na área visando o aprimoramento das
técnicas de proteção vistas nesse trabalho e outras mais.

\nocite{*}
\bibliographystyle{abnt-num}
\bibliography{bibliografia}
\end{document}
