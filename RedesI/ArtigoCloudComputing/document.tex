\documentclass[brazil,12pt]{article}
\usepackage{times}
\usepackage[T1]{fontenc}
\usepackage[utf8]{inputenc}
\makeatletter
\usepackage{babel}
\makeatother
\usepackage[pdfborder={0 0 0}]{hyperref}

\begin{document}

\title{Um Panorama das Técnicas de Segurança em Cloud Computing}
\author{Pedro Paulo Vezzá Campos}
\date{\today}
\maketitle

\begin{abstract}
A Computação em Nuvem, ou cloud computing vem avançando tanto no mercado
empresarial quanto para usuários finais como uma forma de
reduzir custos e facilitar a manutenção ao delegá-la a empresas especializadas.
Porém, ao mesmo tempo surgem diversos questionamentos quanto à segurança dos dados confiados a
empresas terceiras. Neste artigo em formato de \emph{survey} será abordado o
estado da arte nas técnicas de proteção produzidas para cloud computing.  
\end{abstract}

\section{Motivação}
cloud computing é um novo nicho de mercado que tornou-se mais atrativo com o
barateamento de insumos necessários à computação como energia, poder de
processamento, armazenamento e transmissão de dados, permitindo uma economia de
escala \cite{above-clouds}. Através da Computação em nuvem, surgiu a noção de
computação como um serviço, elástico, virtualmente ilimitado e pago apenas pela
porção realmente utilizada, muito similar ao sistema de distribuição elétrica.

Diante desse cenário, \emph{players} como Amazon, Google,
Microsoft, HP, IBM dentre outros, adentraram essa área oferecendo diversas
modalidades de cloud computing, desde a mais abstrata, SaaS ou Software como
Serviço, na qual é fornecido um software pronto para ser utilizado utilizando
recursos da nuvem, até a mais básica, IaaS ou Infraestrutura como Serviço, que
permitem uma personalização da computação desde o nível de \emph{kernel} até
camadas superiores.

Por outro lado, críticas e dúvidas são levantadas por opositores dessa
tecnologia. Um exemplo é Richard Stallman, evangelista do software livre, que
afirma em uma tradução livre:

\begin{quote}
``É estupidez. É pior que estupidez: É uma campanha de marketing. Alguém está
dizendo que isso é inevitável — e sempre que você ouvir alguém falando isso, é
muito provável que seja um conjunto de empresas realizando campanha para
torná-lo verdade.'' \cite{stallman-cloud}
\end{quote}

Paralelamente às oportunidades possibliltadas através do uso de cloud computing
há um aumento das preocupações com a segurança dos dados armazenados na núvem.
Frente a isso diversas pesquisas abordaram técnicas tradicionais e inovadoras
parta solucionar esse problema. Neste trabalho serão apresentadas as mais
importantes dessas propostas apresentadas.

\section{Objetivos}
Neste trabalho serão apresentados inicialmente os conceitos fundamentais da área
de cloud computing como forma de fundamentação para as próximas seções.
Posteriormente, serão apresentados os mais importantes tópicos de preocupação de
usuários e administradores de sistemas a respeito da Computação nas Nuvens.
Por fim, serão apresentados os últimos avanços na área de segurança,
apresentados por diversos pesquisadores da área.

\section{Trabalhos Correlatos}

Em \cite{controlling-data-in-cloud} são apresentados diversas preocupações com
cloud computing, categorizadas segundo a três macro áreas, como apresentado
abaixo. Ao longo desse trabalho serão apresentados e discutidos os itens contidos em cada uma dessas
categorias.
\begin{itemize}
  \item Segurança tradicional
  \item Disponibilidade
  \item Controle de dados por terceiros
\end{itemize}

Em \cite{whats-new-about-cloud-security}, Chen et. al. fazem um paralelo entre
os desafios tradicionais enfrentados pelos provedores de cloud computing e novas
possibilidades ataques. O trabalho é permeado por exemplos de técnicas de
proteção em sistemas históricos, como o Multics, comprometimento de sistemas e
suas consequências, dentre outros.


\section{Conceitos básicos}
O significado de cloud computing ainda é motivo de discussão na indústria e
academia, uma vez que seu significado é amplo. Uma das tentativas de
sistematizar a definição desse termo pode ser encontrada em \cite{above-clouds}.
Ainda, segundo o \emph{National Institute of Standards and Technology}, algumas
das características essenciais do cloud computing incluem:

\begin{itemize}
  \item Serviço sob demanda
  \item Acesso em banda larga
  \item \emph{Pooling} de recursos
  \item Elasticidade rápida
  \item Consumo medido, similar ao do sistema de distribuição elétrica, por
  exemplo.
\end{itemize}

\subsection{Modelos de serviços}
Atualmente há três modelos de serviços reconhecidos como cloud computing, cada
um variando no nível de configurabilidade e abstração, como é possível ver
abaixo:

\begin{description}
  \item[SaaS - Software-as-a-Service] SaaS, ou Software como um Serviço, é a
  modalidade de cloud computing mais abstrata de todas. O usuário possui apenas
  a capacidade de configurar a aplicação utilizada. Nesta faixa encontram-se
  serviços tais como o Gmail, Google Apps, Evernote, etc.
  \item[PaaS - Platform-as-a-Service] PaaS, ou Plataforma como um Serviço, é a
  versão intermediária de computação em núvem. Nesta, o usuário possui o poder
  configurar certas varáveis do ambiente de hospedagem. Aqui encontram-se
  serviços tais como o Google App Engine.
  \item[IaaS - Infrastructure-as-a-Service] IaaS, ou Infraestrutura como um
  Serviço, é a versão mais básica de computação em núvem. Neste ponto o usuário
  tem total controle do ambiente desde o kernel até camadas mais superires. Um
  exemplo de IaaS é o Amazon EC2.
\end{description}

Ainda, há variadas maneiras de implementar uma núvem, variando no
compartilhamento de recursos entre diferentes clientes:

\begin{description}
  \item[Cloud Pública] Nesse modelo qualquer pessoa ou empresa pode contratar o
  serviço de cloud computing.
  \item[Cloud Comunitáira] Aqui o serviço é fornecido a apenas algumas
  organizações.
  \item[Cloud Privada] A núvem passa a ser exclusiva de uma organização.
  \item[Cloud Híbrida] Uma mistura dos modelos anteriores.
\end{description}

\section{Principais preocupações em Cloud Computing}
A natureza da cloud, uma estrutura normalmente externa ao firewall da
organização e possívelmente gerenciada por terceiros, gera a necessidade de
estudos detalhados de análise de riscos antes de adotar uma solução de cloud
computing. Algumas das preocupações estão descritas abaixo:

\begin{description}
  \item[Ataque à virtualização] Os hipervisores de virtualização são pontos
  sensíveis a um ataque direcionado, uma vez que eles são responsáveis por
  garantir o isolamento entre máquinas virtuais. Vulnerabilidades importantes já
  foram encontradas em diversas ferramentas, como apresenta 
  \cite{controlling-data-in-cloud}.
  \item[Vulnerabilidades do provedor de cloud] Outra possibilidade é a invasão
  ser dirigida ao provedor de cloud computing. Um ataque pode comprometer a
  infraestrutura de autenticação do sistema e assim permitir o invasor o acesso
  irrestrito a dados sensíveis do usuário.
  \item[Maior superfície de ataque] Com a cloud encontrando-se possivelmente
  fora do firewall da empresa há a necessidade de haver uma proteção da
  infraestrutura que os conectam.
%   \item[Auditabilidade] 
%   VM-level attack
% Cloud provider vulnerabilities.
% Expanded network attack surface
% Authentication and Authorization
% Auditability.
% Cloud Provider Espionage.
\end{description}

\section{Técnicas tradicionais de segurança}
Na área de técnicas tradicionais, pode-se incluir a criptografia como forma de
controlar o acesso a informações privilegiadas armazenadas na núvem. Ainda, o
fator humano continua sendo crucial, ataques de engenharia social continuam
sendo bastante efetivos em conseguir comprometer um sistema.

Por outro lado, as organizações devem estar preparadas para a recuperação de
disastres, dessa forma, backups são essenciais. A ausência de um cuidado simples
pode levar a sérias consequências como é possível encontrar na bibliografia.


\section{As novas possibilidades de ataques e defesas}
No contexto de cloud computing, uma possibilidade pouco explorada anteriormente
é a de captura do tráfego gerado e a transmissão ativa de dados por um agente
malicioso. Isso é possível graças ao compartilhamento de recursos realizado pela
núvem. Outra possiblidade é a de um invasor conseguir recuperar informações
sobrescritas ou apagadas por um cliente. 

Por outro lado, uma vantagem da computação em núvem é a concentração de
expertise em segurança. Um único provedor de cloud computing pode diluir os
custos para manter políticas de segurança e uma equipe de segurança responsável
por controlar possíveis falhas e repará-las.

Isso leva a uma modalidade de serviços conhecida como
\emph{Security-as-a-Service} na qual um provedor fornece soluções prontas de
segurança tais como frameworks, criptografia, antivírus, etc.

Uma prova da importância da área de segurança em cloud computing é a criação de
eventos importantes na área tais como: ACM Cloud Computing Security Workshop e a
ACM Conference on Computer and Communications Security.

\section{Conclusão}
Como foi possível perceber ao longo do artigo a área de segurança em cloud
computing alia tanto conceitos tradicionais de segurança de redes, quanto novas
técnicas de proteção. A área ainda permanece em constante evolução à medida que
novas técnicas de ataques surgem e novas soluções a essas invasões são
implementadas. A culminação disso, foi o surgimento, por exemplo, do
\emph{Security-as-a-Service} que abrange diversas soluções prontas para serem
aplicadas ao ambiente de computação em núvem.

\nocite{*}
\bibliographystyle{abnt-num}
\bibliography{bibliografia}
\end{document}
