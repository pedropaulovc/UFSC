\documentclass[brazil,12pt]{article}
\usepackage{times}
\usepackage[T1]{fontenc}
\usepackage[utf8]{inputenc}
\makeatletter
\usepackage{babel}
\makeatother
\usepackage[pdfborder={0 0 0}]{hyperref}

\begin{document}

\title{Um Panorama das Técnicas de Segurança em Cloud Computing}
\author{Pedro Paulo Vezzá Campos}
\date{\today}
\maketitle

\begin{abstract}
A Computação em Nuvem, ou Cloud Computing vem avançando tanto no mercado
empresarial quanto para usuários finais como uma forma de
reduzir custos e facilitar a manutenção ao delegá-los a empresas especializadas.
Porém, ao mesmo tempo surgem diversos questionamentos quanto à segurança dos
dados confiados a terceiros. Neste artigo em formato de \emph{survey} será
abordado o estado da arte nas técnicas de proteção produzidas para cloud computing.
\end{abstract}

\section{Motivação}
Cloud computing é um novo nicho de mercado que tornou-se mais atrativo com o
barateamento de insumos necessários à computação como energia, poder de
processamento, armazenamento e transmissão de dados, permitindo uma economia de
escala. \cite{above-clouds} Através da computação em nuvem, surgiu a noção de
computação como um serviço, elástico, virtualmente ilimitado e pago apenas pela
porção realmente utilizada, muito similar ao sistema de distribuição elétrica.

Diante desse cenário, \emph{players} como Amazon, Google,
Microsoft, HP, IBM, dentre outros, adentraram essa área oferecendo diversas
modalidades de cloud computing, desde a mais abstrata, SaaS ou Software como
Serviço, na qual é fornecido um software pronto para ser utilizado utilizando
recursos da nuvem, até a mais básica, IaaS ou Infraestrutura como Serviço, que
permitem uma personalização da computação desde o nível de \emph{kernel} até
camadas superiores.

Por outro lado, críticas e dúvidas são levantadas por opositores dessa
tecnologia. Um exemplo é Richard Stallman, evangelista do software livre, que
afirma em uma tradução livre:

\begin{quote}
``É estupidez. É pior que estupidez: É uma campanha de marketing. Alguém está
dizendo que isso é inevitável — e sempre que você ouvir alguém falando isso, é
muito provável que seja um conjunto de empresas realizando campanha para
torná-lo verdade.'' \cite{stallman-cloud}
\end{quote}

Paralelamente às oportunidades possibliltadas através do uso de cloud computing
há um aumento das preocupações com a segurança dos dados armazenados na nuvem.
Frente a isso diversas pesquisas abordaram técnicas tradicionais e inovadoras
para solucionar esse problema. Neste trabalho serão apresentadas as mais
importantes dessas propostas apresentadas.

Este artigo está organizado da seguinte maneira: A seção 2 deste artigo trata
dos objetivos visados na elaboração desse texto. Na parte 3 há um estudo dos
principais trabalhos correlatos na área de cloud computing. Já na seção 4 são
apresentadosa os principais conceitos relacionados à area, necessários para a
compreenção dos assuntos seguintes. Posteriormente, na seção 5 são apresentadas
as principais preocupações de usuários da nuvem relativos à sua segurança.
Continuando, a parte 6 aborda as principais técnicas tradicionais de segurança,
aplicáveis à cloud. Ainda, a parte 7 explora as novas possibilidades de ataques
e defesas originados do uso (e abuso) da computação em nuvem. Por fim, a parte 8
apresenta as conclusões mais importantes desse trabalho.

\section{Objetivos}
O principal objetivo deste \emph{survey} é apresentar um panorama geral das
técnicas de segurança em cloud computing, apresentando tanto ataques possíveis a
um usuário ou gerente da nuvem quanto maneiras de proteger-se de tais invasões.
Para isso, serão apresentados inicialmente os conceitos principais da área
de cloud computing como forma de fundamentação. Posteriormente, serão
apresentados os mais importantes tópicos de preocupação de usuários e
administradores de sistemas a respeito da computação nas nuvens. Por fim, será
apresentado um paralelo entre técnicas de ataque e defesa existentes.

\section{Trabalhos Correlatos}

Em \cite{controlling-data-in-cloud}, Chow et. al. apresentam diversas
preocupações com cloud computing, categorizadas segundo três macro áreas, como
apresentado abaixo. Ao longo desse trabalho serão apresentados e discutidos os
itens contidos em cada uma dessas categorias.
\begin{itemize}
  \item Segurança tradicional
  \item Disponibilidade
  \item Controle de dados por terceiros
\end{itemize}

Em \cite{whats-new-about-cloud-security}, Chen et. al. fazem um paralelo entre
os desafios tradicionais enfrentados pelos provedores de cloud computing e novas
possibilidades ataques. O trabalho é permeado por exemplos de técnicas de
proteção em sistemas históricos, como o Multics, comprometimento de sistemas e
suas consequências, dentre outros.

Ristenpart et. al. apresentam em \cite{hey-you-get-off-of-my-cloud} um
exemplo prático de uma tentativa de invasão em um serviço de computação em
nuvem, tomando como exemplo o Amazon EC2. No trabalho os autores apresentam
detalhadamente as técnicas empregadas para mapear a rede que compõe a nuvem da
empresa, heurísticas adotadas para conseguir que uma máquina virtual invasora
seja instalada na mesma máquina física que a máquina virtual alvo e, por fim,
explorando vulnerabilidades da ferramenta de virtualização conseguir gerar um
canal lateral entre máquinas virtuais, burlando o isolamento imposto pelo
\emph{hipervisor}. 

Jensen, et. al. selecionaram em \cite{technical-security-issues} diversas
questões relacionadas à segurança no contexto da computação em nuvem. Seu
enfoque principal é voltado ao estudo das técnicas de segurança empregadas em
protocolos de serviços Web e seus frameworks, tais como SOAP e WS-Security, e
também às abordagens adotadas por navegadores Web. Uma conclusão derivada de
suas observações é a necessidade de por um lado fortalecer as técnicas de
segurança empregadas nessas duas áreas e por outro prover a integração de ambas
para fortalecer a segurança no uso de aplicações baseadas na nuvem.

\section{Conceitos básicos}
O significado de cloud computing ainda é motivo de discussão na indústria e
academia, uma vez que seu significado é amplo. Uma das tentativas de
sistematizar a definição desse termo pode ser encontrada em \cite{above-clouds}.
Ainda, segundo o \emph{National Institute of Standards and Technology}, algumas
das características essenciais do cloud computing incluem
\cite{nist-definition-cloud-computing}:

\begin{itemize}
  \item Serviço sob demanda
  \item Acesso em banda larga
  \item \emph{Pooling} de recursos
  \item Elasticidade rápida
  \item Consumo medido, similar ao do sistema de distribuição elétrica, por
  exemplo.
\end{itemize}

\subsection{Modelos de serviços}
Atualmente há três modelos de serviços reconhecidos como cloud computing, cada
um variando no nível de configurabilidade e abstração, como é possível ver
abaixo:

\begin{description}
  \item[SaaS - Software-as-a-Service] SaaS, ou Software como um Serviço, é a
  modalidade de cloud computing mais abstrata de todas. O usuário possui apenas
  a capacidade de configurar a aplicação utilizada. Nesta faixa encontram-se
  serviços tais como o Gmail, Google Apps, Evernote, etc.
  \item[PaaS - Platform-as-a-Service] PaaS, ou Plataforma como um Serviço, é a
  versão intermediária de computação em nuvem. Nesta, o usuário possui o poder
  configurar certas varáveis do ambiente de hospedagem. Aqui encontram-se
  serviços tais como o Google App Engine.
  \item[IaaS - Infrastructure-as-a-Service] IaaS, ou Infraestrutura como um
  Serviço, é a versão mais básica de computação em nuvem. Neste ponto o usuário
  tem total controle do ambiente desde o kernel até camadas mais superires. Um
  exemplo de IaaS é o Amazon EC2.
\end{description}

Ainda, há variadas maneiras de implementar uma nuvem, variando no
compartilhamento de recursos entre diferentes clientes:

\begin{description}
  \item[Cloud Pública] Nesse modelo qualquer pessoa ou empresa pode contratar o
  serviço de cloud computing.
  \item[Cloud Comunitáira] Aqui o serviço é fornecido a apenas algumas
  organizações.
  \item[Cloud Privada] A nuvem passa a ser exclusiva de uma organização.
  \item[Cloud Híbrida] Uma mistura dos modelos anteriores.
\end{description}

\section{Principais preocupações}
A natureza da nuvem, uma estrutura normalmente externa ao \emph{firewall} da
organização e possívelmente gerenciada por terceiros, gera a necessidade de
estudos detalhados de análise de riscos antes de adotar uma solução de cloud
computing. Algumas das preocupações apresentadas por
\cite{controlling-data-in-cloud} estão descritas abaixo:

\subsection{Segurança tradicional}
\begin{description}
  \item[Ataque à virtualização] Os hipervisores de virtualização são pontos
  sensíveis a um ataque direcionado, uma vez que eles são responsáveis por
  garantir o isolamento entre máquinas virtuais. Vulnerabilidades importantes já
  foram encontradas em diversas ferramentas, como apresenta 
  \cite{controlling-data-in-cloud}.
  \item[Maior superfície de ataque] Com a cloud encontrando-se possivelmente
  fora do firewall da empresa há a necessidade de haver uma proteção da
  infraestrutura, principalmente a rede que os conectam.
  \item[Vulnerabilidades do provedor de cloud] Outra possibilidade é a invasão
  ser dirigida ao provedor de cloud computing. Um ataque pode comprometer a
  infraestrutura de autenticação do sistema e assim permitir o invasor o acesso
  irrestrito a dados sensíveis do usuário.
  \item[Deficiências de protocolos] Serviços baseados em cloud computing
  dependem fortemente de protocolos de serviços web, tais como o SOAP, para a
  troca de informação. Uma vulnerabilidade nesse ponto da interação entre
  cliente e servidor permite um invasor realizar operações arbitrárias no
  serviço fazendo-se passar por um usuário legítimo. O serviço Amazon EC2 foi
  vítima de um ataque conhecido como \emph{wrapping attack} em 2008 devido a
  vunerabilidades do XML Signature já conhecidas em 2005.
  \cite{technical-security-issues}
  \item[Vulnerabilidades na infraestrutura] Em 2008 foi descoberta a
  possiblidade de ``envenenamento'' de caches DNS com dados inválidos ou
  maliciosos. Isso tornou o protocolo inseguro, criando a necessidade de uso de
  uma variante segura, tal como o DNSSEC, e ainda medidas extra de segurança no
  protocolo de transporte. Outro ponto sensível de ataque são os roteadores,
  switches e access-points domésticos, dispositivos raramente atualizados e
  corretamente configurados. Tudo isso leva a uma redução da confiabilidade dos
  dados recebidos de uma URL sem um devido tratamento de segurança adequado por
  outros meios.
\end{description}

\subsection{Disponibilidade}
\begin{description}
  \item[Controle da integridade computacional] Ao contratar um serviço de cloud
  computing o cliente espera que o fornecedor da cloud aja de boa-fé, fornecendo
  valores de computação corretos. Dependendo do nível de criticidade de um
  serviço localizado na nuvem, apenas essa promessa é insuficiente. Para
  resolver esse problema pode-se adotar a redundância, por exemplo. O projeto
  Folding@Home envia tarefas idênticas para computadores diferentes no intuito
  de atingir um consenso nos resultados obtidos. A Google, por outro lado,
  armazena cópias de dados em diversas máquinas diferentes como forma de
  tolerância a falhas.
  \item[Estabilidade] Dependendo do nível de criticidade de uma aplicação
  localizada na nuvem, um contrato garantindo uma disponibilidade
  pré-estabelecida é necessária. Porém, mesmo com diversas medidas de
  redundância de dados, enlaces, energia, dentre outros, certas vezes um
  serviço pode não cumprir o acordo, gerando prejuízos e transtornos aos
  usuários. Serviços famosos tais como o Gmail e o Amazon EC2 já sofreram com
  indisponibilidades de 24 e 7 horas respectivamente. \cite{controlling-data-in-cloud}
\end{description}


\subsection{Controle de dados por terceiros}
\begin{description}
  \item[Aprisionamento de dados] Uma das grandes preocupações com o cloud
  computing é que um cliente passe a ser dependente de um fornecedor único (ou
  um pequeno grupo) de forma que o primeiro passe a ser um refém do segundo.
  Esse problema pode surgir através da imposição do uso de formatos ou APIs
  proprietárias. O Google App Engine, por exemplo, impõe o uso de tecnologias
  internas à Google, como o BigTable, GFS, dentre outros. Claramente uma solução
  a esse problema é a adoção de padrões livres. Uma amostra dessa tentativa é a
  API GoGrid.
  \item[Espionagem] Usuários e empresas são muito receosos em fornecer seus
  dados particulares a terceiros. Um de seus principais medos é o roubo de
  informação sensível pelo fornecedor dos serviços de cloud computing. Um
  exemplo disso é o Gmail que logo após seu lançamento foi bastante criticado
  por utilizar \emph{crawlers} para ler os emails dos usuários a fim de fornecer
  anúncios contextualizados. Porém, passados 6 anos de seu lançamento tais
  temores amainaram.
  \item[Auditabilidade] Ao delegar tarefas a uma empresa fornecedora de serviços
  de cloud computing, automaticamente há uma diminuição do controle sobre a
  manipulação de seus dados. Como consequência, diminui-se também a capacidade
  de auditar as operações em busca de problemas. Dada a natureza dinâmica da
  computação em nuvem torna-se bastante difícil garantir o isolamento dos dados
  de outros clientes. Atualmente a auditabilidade ocorre através de documentação
  e auditorias manuais. \cite{controlling-data-in-cloud}
  \item[Apreensão de equipamentos] Diante de uma possível de prática de crime
  por algum cliente de serviço de cloud computing um juiz pode determinar um
  mandado de busca e apreensão de equipamentos utilizados pelo réu. Como vários
  clientes coexistem em uma mesma máquina há um risco de que dados de usuários
  sem relação com o réu sejam recolhidos, podendo não serem retornados após o
  fim das investigações.
\end{description}

\section{Técnicas tradicionais}
Na área de técnicas tradicionais de segurança, pode-se incluir a criptografia
como forma de controlar o acesso a informações privilegiadas armazenadas na
nuvem. O uso de criptografia apresentava um bom nível de segurança até o
surgimento em 2004 dos primeiros ataques de \emph{phishing}. O \emph{phishing} é
uma técnica de invasão que consiste em enganar o usuário de um serviço web,
comumente através de emails falsos, ataques ao DNS, ou nomes de domínio
semelhantes, fazendo-o fornecer suas credenciais, e/ou dados pessoais.
\cite{technical-security-issues}

Neste ataque, o invasor aproveita-se da confiança cedida pelo usuário a algum
serviço conhecido para roubar seus dados. Em uma versão mais elaborada, o agente
malicioso pode possuir uma autenticação criptográfica como forma de ludibriar
usuários mais cautelosos. Um resultado disso é a gradual substituição do icônico
cadeado das interfaces dos navegadores web, um símbolo de falsa segurança,
por outras metáforas mais apropriadas.

O fator humano ainda é explorado de várias maneiras com sucesso. Uma prova disso
é que ataques de engenharia social continuam sendo bastante efetivos em
conseguir comprometer sistemas.

Um exemplo dessa última afirmação foi apresentada na conferência BlackHat USA
2009. Invasores tiveram acesso a máquinas virtuais instaladas no Amazon EC2
apenas fornecendo imagens de sistemas operacionais infectadas de maneira
aparentemente oficial, utilizando nomes como ``fedora\_core''.
\cite{whats-new-about-cloud-security}

Por outro lado, as organizações devem estar preparadas para a recuperação de
desastres, dessa forma, backups são essenciais. A ausência de um cuidado simples
pode levar a sérias consequências como é possível encontrar na bibliografia.

No Brasil um caso recente foi o do migre.me: Uma falha catastrófica no servidor
que mantinha o serviço causou não só a perda de todos o banco de 
redirecionamentos quanto os backups do site. Como agravante, o administrador 
não mantinha backups próprios dos dados do site, apenas do código. A
consequência disso é que milhares de URLs do migre.me tornaram-se
irrecuperáveis, causando diversos transtornos e possivelmente prejuízos aos
usuários do serviço.


\section{As novas possibilidades}
Uma prova da importância da área de segurança voltada para o cloud computing
é a criação de eventos importantes na área tais como: ACM Cloud Computing
Security Workshop e a ACM Conference on Computer and Communications Security.
Nelas são discutidas diferentes abordagens para aprimorar as defesas dos
serviços contra ataques cada vez mais sofisticados.

Nesta seção serão apresentados dois tópicos um mostrando as últimas técnicas de
ataques ao cloud computing e outro apresentando o estado da arte
das técnicas de defesa na área.

\subsection{Novos ataques}
O nível de virtualização que a computação em nuvem traz possibilita diversos
novos ataques. Apesar de ainda não haver notícia de alguma grande invasão a
usuários de cloud computing, já há provas de conceitos disponíveis, tal como a
decrita em \cite{hey-you-get-off-of-my-cloud}.

Primeiramente através de técnicas tais como o mapeamento da rede da empresa
fornecedora da nuvem há a possibilidade de um agente malicioso conseguir
induzir o sistema de alocação de máquinas virtuais a instalar uma máquina
virtual invasora na mesma máquina física que a máquina virtual alvo. Com isso,
diversos dos temores elencados anteriormente com respeito a cloud computing
tornam-se relevantes.

Neste momento, o invasor pode realizar uma escuta passiva em diversas partes da
máquina física com o intuito de capturar informações sensíveis de seu alvo.
Pode-se varrer blocos livres do disco rígido em busca de dados removidos
anteriormente removidos mas não destruidos apropriadamente, ou ainda varrer
buffers de rede, dados armazenados em caches, dentre outros. Outra abordagem
possível é explorar uma baixa entropia em números aleatórios gerados para
recuperar chaves criptográficas.

Elevando o nível de sofisticação do ataque, pode-se explorar vulnerabilidades da
ferramenta de virtualização para gerar um canal lateral entre máquinas
virtuais, burlando o isolamento imposto pelo \emph{hipervisor}. Vulnerabilidades
em diferentes ferramentas, tais como XEN e VMware já foram encontradas,
permitindo invasões dessa forma. \cite{controlling-data-in-cloud}

Ainda, com a concentração de vários clientes diferentes em somente um local, um
ataque massivo pode afetar diversos usuários ao mesmo tempo. Um caso
exemplificável foi o \emph{blacklisting} de uma grande faixa de endereços IP do
Amazon EC2 para o envio de emails depois que spammers conseguiram subverter as
proteções levantadas pela empresa para evitar abusos do serviço. Vários usuários
sofreram interrupção na entrega normal de seus emails. Como medida para corrigir
o problema, agora os clientes devem preencher um formulário junto à Amazon
informando a necessidade de envio de grandes quantidades de emails para que seja
requisitado o \emph{whitelisting} da faixa de IPs correspondente àquele usuário
específico. \cite{above-clouds} \cite{whats-new-about-cloud-security}

\subsection{Novas defesas}
Diante das ameaças discutidas anteriormente, há algumas possibilidades de
solução, algumas paliativas, outras mais efetivas. Primeiramente, um fornecedor
de serviços de cloud computing pode impedir ou pelo menos desacelerar um invasor
ao obfuscar tanto a estrutura interna de seus serviços quanto a política de
alocação de máquinas virtuais, além de impedir testes simples de co-residência
no intuito de um invasor instalar-se na mesma máquina fisica que seu alvo.

Ainda, com o aperfeiçoamento dos hipervisores, é possível empregar mais medidas
de mascaramento para minimizar a quantidade de informação que pode ser extraída
através de canais laterais. Porém, este ponto possui como fraqueza o fato que
repetidamente novas falhas de segurança são encontradas, abrindo novamente
margem para um agente malicioso realizar invasões a seus alvos.

Por outro lado, uma crescente vantagem da computação em nuvem é a concentração
de expertise em segurança. Um único provedor de cloud computing pode diluir os
custos para manter políticas de segurança e uma equipe de segurança responsável
por controlar possíveis falhas e repará-las.

Isso leva a uma modalidade de serviços conhecida como
\emph{Security-as-a-Service} na qual um provedor fornece soluções prontas de
segurança tais como frameworks, criptografia, antivírus, etc.

Contudo, talvez a solução mais eficaz para os problemas da computação em nuvem é
a transparência da empresa em admitir os riscos envolvidos, apresentando opções
diferentes adequadas às necessidades de cada usuário. Clientes interessados em
soluções de cloud computing mas mais preocupados com seus dados podem aceitar
pagar um valor extra pela garantia que as máquinas físicas utilizadas por eles
contenham apenas máquinas virtuais de sua responsabilidade, mesmo que isso
incorra em uma subutilização de recursos.

Dado que os custos para fornecer essa possibilidade de escolha sejam
relativamente pequenos, no máximo o preço para alugar um servidor dedicado, por
exemplo, mesmo grandes clientes podem fazer uso dessa tecnologia sem fortes
penalidades de gastos diante dos custos totais. Assim, garante-se um serviço
diferenciado e com uma prova palpável de maior segurança a clientes com
necessidade especiais de proteção de dados.


\section{Conclusão}
Como foi possível perceber ao longo do artigo a área de segurança em cloud
computing alia tanto conceitos tradicionais de segurança de redes, quanto novas
técnicas de proteção. Ela ainda permanece em constante evolução à medida que
novas técnicas de ataques surgem e novas soluções a essas invasões são
implementadas.

Através de diferentes exemplos práticos envolvendo diversas empresas
fornecedoras de serviços de computação em nuvem foi possível ver as
consequências severas que podem surgir a partir de vulnerabilidades e descuidos
tanto dos administradores da cloud quanto dos usuários dos serviços. Isso
corrobora a importância das pesquisas na área visando o aprimoramento das
técnicas de proteção vistas nesse trabalho e outras mais.

\nocite{*}
\bibliographystyle{abnt-num}
\bibliography{bibliografia}
\end{document}
