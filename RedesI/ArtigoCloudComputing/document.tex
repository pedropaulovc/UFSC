\documentclass[brazil,12pt]{article}
\usepackage{times}
\usepackage[T1]{fontenc}
\usepackage[utf8]{inputenc}
\makeatletter
\usepackage{babel}
\makeatother
\usepackage[pdfborder={0 0 0}]{hyperref}

\begin{document}

\title{Um Panorama das Técnicas de Segurança em Cloud Computing}
\author{Pedro Paulo Vezzá Campos}
\date{\today}
\maketitle

\begin{abstract}
A Computação em Nuvem, ou Cloud Computing vem avançando tanto no mercado
empresarial quanto para usuários finais como uma forma de
reduzir custos e facilitar a manutenção ao delegá-la a empresas especializadas.
Porém, ao mesmo tempo surgem diversos questionamentos quanto à segurança dos dados confiados a
empresas terceiras. Neste artigo em formato de \emph{survey} será abordado o
estado da arte nas técnicas de proteção produzidas para Cloud Computing.  
\end{abstract}

\section{Motivação}
Cloud Computing é um novo nicho de mercado que tornou-se mais atrativo com o
barateamento de insumos necessários à computação como energia, poder de
processamento, armazenamento e transmissão de dados, permitindo uma economia de
escala \cite{above-clouds}. Através da Computação em nuvem, surgiu a noção de
computação como um serviço, elástico, virtualmente ilimitado e pago apenas pela
porção realmente utilizada, muito similar ao sistema de distribuição elétrica.

Diante desse cenário, \emph{players} como Amazon, Google,
Microsoft, HP, IBM dentre outros, adentraram essa área oferecendo diversas
modalidades de Cloud Computing, desde a mais abstrata, SaaS ou Software como
Serviço, na qual é fornecido um software pronto para ser utilizado utilizando
recursos da nuvem, até a mais básica, IaaS ou Infraestrutura como Serviço, que
permitem uma personalização da computação desde o nível de \emph{kernel} até
camadas superiores.

Por outro lado, críticas e dúvidas são levantadas por opositores dessa
tecnologia. Um exemplo é Richard Stallman, evangelista do software livre, que
afirma em uma tradução livre:

\begin{quote}
``É estupidez. É pior que estupidez: É uma campanha de marketing. Alguém está
dizendo que isso é inevitável — e sempre que você ouvir alguém falando isso, é
muito provável que seja um conjunto de empresas realizando campanha para
torná-lo verdade.'' \cite{stallman-cloud}
\end{quote}

Em \cite{controlling-data-in-cloud} são apresentados diversas preocupações com
Cloud Computing, categorizadas segundo a listagem abaixo. Ao longo desse
trabalho serão apresentados e discutidos os itens contidos em cada uma dessas
categorias.
\begin{itemize}
  \item Segurança tradicional
  \item Disponibilidade
  \item Controle de dados por terceiros
\end{itemize}

\section{Objetivos}
Neste trabalho serão apresentados inicialmente os conceitos fundamentais da área
de Cloud Computing como forma de fundamentação para as próximas seções.
Posteriormente, serão apresentados os mais importantes tópicos de preocupação de
usuários e administradores de sistemas a respeito da Computação nas Nuvens.
Por fim, serão apresentados os últimos avanços na área de segurança,
apresentados por diversos pesquisadores da área.

\section{Trabalhos Correlatos}

Em \cite{whats-new-about-cloud-security} há uma comparação com técnicas antigas
e recentes de proteção em sistemas computacionais

-- Cont --

Já em \cite{controlling-data-in-cloud} os autores discorrem elencando 


\section{Conceitos básicos}

\section{Técnicas tradicionais de segurança}

\section{As novas possibilidades de ataques e defesas}



security as a service -> frameworks para segurança, criptografia
\nocite{*}
\bibliographystyle{abnt-num}
\bibliography{bibliografia}
\end{document}
