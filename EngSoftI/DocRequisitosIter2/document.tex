\documentclass[brazil,times]{abnt}
\usepackage[T1]{fontenc}
\usepackage[utf8]{inputenc}
\makeatletter
\usepackage{babel}
\makeatother
\usepackage{graphicx}
\usepackage[pdfborder={0 0 0}]{hyperref}
\begin{document}

\autor{Pedro Paulo Vezzá Campos \\ Rafael Elias Pedretti
\\ Juarez Angelo Piazza Sacenti}

\titulo{Documento de Requisitos - Iteração 2}

\comentario{Trabalho apresentado para avaliação na disciplina INE5417, do curso
de Bacharelado em Ciências da Computação, turma 04208, da Universidade Federal
de Santa Catarina, ministrada pela professora Patrícia Vilain}

\instituicao{Departamento de Informática e Estatística \par Centro
Tecnológico\par Universidade Federal de Santa Catarina}

\local{Florianópolis - SC, Brasil}

\data{\today}

\capa

\folhaderosto

\tableofcontents

\chapter{Levantamento de Requisitos}
\section{Requisitos funcionais}
\subsection{Diagrama de Casos de Uso}


\subsection{Editar usuário}


\subsection{Editar Estória/Task}


\subsection{Acessar o histórico do desenvolvimento}


\subsection{Acessar árvore de dependências de uma task}


\subsection{Mostrar para o usuário quais são tasks prioritárias}



\chapter{Análise}
\section{Diagrama de Classes Conceituais}


\chapter{Projeto}
\section{Diagrama de Classes}


\section{Diagramas de Interação}
\subsection{Editar usuário}


\subsection{Editar Estória/Task}


\subsection{Acessar o histórico do desenvolvimento}


\subsection{Acessar árvore de dependências de uma task}


\subsection{Mostrar para o usuário quais são tasks prioritárias}


\section{Projeto da Persistência dos Objetos}


\section{Diagrama de Atividades}
\subsection{Diagrama de Atividades 1}


\subsection{Diagrama de Atividades 2}


\section{Statecharts}
\subsection{Diagrama de Atividades 1}


\subsection{Diagrama de Atividades 2}



\end{document}