\documentclass[brazil,times]{abnt}
\usepackage[T1]{fontenc}
\usepackage[utf8]{inputenc}
\usepackage{hyperref}
\makeatletter
\usepackage{babel}
\makeatother
\begin{document}

\autor{Pedro Paulo Vezzá Campos \\ Rafael Elias Pedretti
\\ Juarez Angelo Piazza Sacenti}

\titulo{Documento de Requisitos - Iteração 1}

\comentario{Trabalho apresentado para avaliação na disciplina INE5417, do curso
de Bacharelado em Ciências da Computação, turma 04208, da Universidade Federal
de Santa Catarina, ministrada pela professora Patrícia Vilain}

\instituicao{Departamento de Informática e Estatística \par Centro
Tecnológico\par Universidade Federal de Santa Catarina}

\local{Florianópolis - SC, Brasil}

\data{25 de agosto de 2010}

\capa

\folhaderosto

\tableofcontents

\chapter{Perfil do Cliente e Visão do Problema}
O LabUFSC é um dos maiores maiores laboratórios de informática da UFSC, com um
parque de 194 máquinas. Como forma de melhorar o serviço de 
suporte e manutenção dos equipamentos o laboratório demanda de um software de
help desk.

Espera-se do software que ele seja capaz de permitir que um usuário registre um
pedido de suporte através de uma interface denominada “Front-end”. Tal pedido
será registrado em uma interface restrita à equipe de manutenção, o “Back-end”,
que poderá tratar o problema de diferentes maneiras: Delegação (À equipe da
secretaria do laboratório), resolução remota ou presencial. Ao final do
processo um administrador poderá marcar um problema como resolvido ou não,
adicionando informações relevantes ao problema.

O sistema deverá armazenar as ocorrências para consulta posterior pela equipe
de manutenção. Deve ser possível a geração de relatórios diversos, tais como a
evolução do número de pedidos, resoluções, tempo médio de resposta, dentre
outros.

Há, ainda, a possibilidade de um administrador configurar o sistema para um
modo automático de operação, que consiste em armazenar e redirecionar os
pedidos de suporte para a equipe da secretaria, que tomará apenas consentimento
do problema através da interface de “Back-end Secretaria”, podendo ou não resolvê-lo.

O ecossistema é heterogêneo, composto por diversos sistemas operacionais, sendo
necessário um sistema multi-plataforma (Linux e Windows).

Não há restrições de tempo rígidas, sendo tolerável tempos de resposta do
sistema (excetuado tempo de transmissão da rede) de aproximadamente 1 segundo.

\chapter{Requisitos Funcionais}
\section{Front-end: Reportar Problema}
\begin{description}
\item[Escopo:] Sistema Front-end
\item[Nível:] Meta do usuário
\item[Ator Primário:] Usuário

\item[Stakeholders e seus interesses:] \hfill \\ 
Usuário: Deseja que seu pedido de suporte seja recebido e atendido

\item[Pré-condições:] O sistema e a rede estão operantes. O aluno possui
matrícula válida (segundo regras descritas na Especificação Suplementar).

\item[Pós-condições:] É emitida uma notificação para o usuário.

\item[Fluxo Básico ou Cenário Principal:] \hfill
\begin{enumerate}
  \item O usuário acessa o Front-end
  \item O usuário informa sua matrícula e uma breve descrição do seu problema.
  \item O usuário submete as informações.
  \item O sistema informa o usuário do sucesso do envio.
  \item O usuário é atendido e seu problema é resolvido.
\end{enumerate}

\item[Extensões:] \hfill
\begin{description}
	\item[*a.] A qualquer momento, o sistema ou a rede falha:
		\begin{enumerate}
 			\item O sistema informa o usuário da incapacidade de envio e sugere chamar    
  			pessoalmente um atendente.
  			\item  O sistema tenta periodicamente reestabelecer a conexão.
		\end{enumerate}

	\item[2a.] Matrícula inválida ou descrição em branco:
	\begin{enumerate}
		\item  O sistema sinaliza o erro e rejeita a entrada.
	\end{enumerate}
\end{description}
\item[Requisitos especiais:] Nenhum

\item[Tecnologia:] \hfill
\begin{description} 
	\item[2a.] A entrada dos dados é feita pelo teclado.
	\item[4a.] O sistema informa o usuário por meio do monitor.
\end{description}
\item[Freqüência de Ocorrência:] Raramente.

\end{description}



\section{Back-end: Notificar Problema}
\begin{description}
\item[Escopo:] Sistema Back-end
\item[Nível:] Meta do Administrador
\item[Ator Primário:] Administrador
\item[Stakeholders e seus interesses:] \hfill \\
Administrador: Deseja ser notificado de novos pedidos e que os pedidos
sejam registrados.
\item[Pré-condições:] O sistema e a rede estão operantes. Há, pelo menos, um pedido 
pendente.
\item[Pós-condições:]  É emitida uma notificação ao administrador e o pedido é        
registrado no Back-end.
\item[Fluxo Básico ou Cenário Principal:]\hfill
\begin{enumerate}
  \item O sistema notifica o Administrador de pedidos pendentes.
  \item O administrador analisa o problema e aciona o sistema de resolução
  remota.
  \item O problema é resolvido pelo administrador.
  \item O administrador marca o problema como resolvido.
  \item O administrador registra a solução.
  \item \emph{Repete os passos de 1 a 5 enquanto houver pedidos pendentes.}
\end{enumerate}

\item[Extensões:]\hfill
\begin{description}
	\item[*a.] A qualquer momento, o sistema ou a rede falha: \hfill
	\begin{enumerate}
 		\item A rede deve ser reestabelecida e o sistema
 		reiniciado manualmente.
	\end{enumerate} 

	\item[2a.] O administrador decide fazer atendimento presencial:
	\begin{enumerate}
  		\item O administrador resolve localmente o problema.
	\end{enumerate}

	\item[2b.] O administrador delega o problema à secretaria:
	\begin{enumerate}
  		\item A secretaria é notificada do problema.
	\end{enumerate}

	\item[3a.] O problema é resolvido pela Secretaria:
	\begin{enumerate}
  		\item A secretaria notifica o administrador da resolução do problema.
	\end{enumerate}

	\item[3b.] O problema não é resolvido pela Secretaria:
	\begin{enumerate}
  		\item A secretaria notifica o administrador da não resolução do problema.
	\end{enumerate}

	\item[(4-5)a.] O problema não foi resolvido:
	\begin{enumerate}
  		\item O problema continua como pendente até ser resolvido.
	\end{enumerate}
\end{description}

\item[Requisitos especiais:] Dispositivo de aviso sonoro.
\item[Tecnologia:] Nenhuma.
\item[Freqüência de Ocorrência:] Regular.

\end{description}
\section{Back-end: Gerar Relatórios}
\begin{description}
\item[Escopo:] Sistema Back-end
\item[Nível:] Meta do Administrador
\item[Ator Primário:] Administrador
\item[Stakeholders e seus interesses:] \hfill \\
Administrador: Deseja obter de maneira fácil relatórios que permitam coletar
estatísticas sobre os pedidos de suporte.

\item[Pré-condições:] O sistema está operante e o Administrador realizou o \emph{login}
no Back-end
\item[Pós-condições:]  É gerado o relatório requisitado pelo Administrador.
\item[Fluxo Básico ou Cenário Principal:]\hfill
\begin{enumerate}
  \item O Administrador informa o relatório desejado. As possibilidades de
  relatórios são: Número de pedidos, número de resoluções, tempo médio de
  resposta.
  \item O Back-end gerará o relatório pedido tendo como período de coleta os
  últimos 30 dias.
\end{enumerate}

\item[Extensões:]\hfill
\begin{description}
	\item[*a.] A qualquer momento, o sistema falha: \hfill
	\begin{enumerate}
 		\item O sistema deve ser reiniciado manualmente.
 		\item O Administrador deve informar suas credenciais novamente.
	\end{enumerate} 
\end{description}
\item[Requisitos especiais:] Nenhum.
\item[Tecnologia:] O relatório será gerado em formato de texto puro.
\item[Freqüência de Ocorrência:] Raramente.

\end{description}



\section{Back-end: Delegar Problema}
\begin{description}
\item[Escopo:] Sistema Back-end
\item[Nível:] Meta do Administrador
\item[Ator Primário:] Administrador
\item[Stakeholders e seus interesses:] \hfill \\
Administrador: Deseja delegar uma resolução de um pedido pendente à Secretaria.

\item[Pré-condições:] O sistema e a rede estão operantes e o Administrador
realizou o \emph{login} no Back-end. Há pelo menos um pedido
pendente.

\item[Pós-condições:] O Administrador confirmou a solução do problema e encerrou
o pedido.
\item[Fluxo Básico ou Cenário Principal:]\hfill
\begin{enumerate}
  \item O Administrador seleciona e envia o pedido pendente a ser delegado à
  Secretaria.
  \item \emph{Há a execução do caso de uso descrito na seção
  \ref{caso-receber-delegacao}}
  \item O Administrador recebe a resposta enviada por um Atendente através do
  Back-end Secretaria.
  \item O Administrador confirma a solução do pedido e encerra-o.
  \item O Back-end deve armazenar a confirmação de recebimento, a resposta
  do problema proposta e o atendente que resolveu o problema. O pedido de
  suporte permanece pendente até um Administrador confirmar a solução do Atendente.
\end{enumerate}

\item[Extensões:]\hfill
\begin{description}
	\item[1a.] A delegação não foi enviada para a Secretaria.
	\begin{enumerate}
		\item O Administrador é notificado da impossibilidade de envio.
		\item O problema permanece como pendente.
	\end{enumerate}

	\item[2a.] A delegação não foi respondida pela Secretaria.
	\begin{enumerate}
 		\item O problema permanece como pendente.
	\end{enumerate}
		
	\item[4a.] O Administrador não confirma a resolução do problema. 
	\begin{enumerate}
 		\item O problema permanece como pendente.
	\end{enumerate} 
\end{description}
\item[Requisitos especiais:] Nenhum.
\item[Tecnologia:] Nenhuma.
\item[Freqüência de Ocorrência:] Regular.
\end{description}



\section{Back-end Secretaria: Receber Delegação \label{caso-receber-delegacao}}
\begin{description}
\item[Escopo:] Sistema Back-end Secretaria
\item[Nível:] Meta do Atendente
\item[Ator Primário:] Atendente
\item[Stakeholders e seus interesses:] \hfill \\
Atendente: Deseja ser notificado quando houver um pedido pendente delegado por
um Administrador.

\item[Pré-condições:] O sistema e a rede estão operantes.
\item[Pós-condições:] Um atendente recebeu um notificação de problema referente
ao pedido pendente.
\item[Fluxo Básico ou Cenário Principal:]\hfill
\begin{enumerate}
  \item O atendente recebe o pedido. O problema é solucionado e o atendente
  informa a resolução.
  \item O Atendente fornece suas credenciais e envia a resposta do problema.
  \item O Back-end deve armazenar a confirmação de recebimento, a resposta
  do problema proposta e o atendente que resolveu o problema.
\end{enumerate}

\item[Extensões:]\hfill
\begin{description}
	\item[2a.] O Atendente não foi capaz de solucionar o problema.
	\begin{enumerate}
 		\item O Atendente informa a incapacidade de resolução.
	\end{enumerate} 
	
	\item[3a.] O Atendente fornece credenciais inválidas. 
	\begin{enumerate}
 		\item O Back-end Secretaria pede novamente as credenciais.
 		\item \emph{O passo anterior é repetido até que haja credenciais válidas.}
	\end{enumerate} 
\end{description}
\item[Requisitos especiais:] Nenhum.
\item[Tecnologia:] Nenhuma.
\item[Freqüência de Ocorrência:] Regular.

\end{description}



\section{Back-end: Adicionar Usuários}
\begin{description}
\item[Escopo:] Sistema Back-end
\item[Nível:] Meta do Administrador
\item[Ator Primário:] Administrador
\item[Stakeholders e seus interesses:] \hfill \\
Administrador: Deseja adicionar administradores ou atendentes.

\item[Pré-condições:] O sistema está operante. O Administrador está logado.
\item[Pós-condições:] O usuário foi adicionado.
\item[Fluxo Básico ou Cenário Principal:]\hfill
\begin{enumerate}
  \item O Administador fornece o login e senha do usuário a ser criado.
  \item O Administrador fornece o grupo do novo usuário (Administrador ou
  Atendente)
  \item A operação é executada.
\end{enumerate}

\item[Extensões:]\hfill
\begin{description}
	\item[1a.] O usuário novo já existe.
	\begin{enumerate}
 		\item O Back-end informa ao Administrador que o usuário já existe.
 		\item O Back-end requisita um novo login.
 		\item \emph{O passo anterior repete-se enquanto o usuário fornecido já
 		existir.}
	\end{enumerate}

\end{description}
\item[Requisitos especiais:] Nenhum.
\item[Tecnologia:] Nenhuma.
\item[Freqüência de Ocorrência:] Raramente.

\end{description}



\section{Back-end: Remover Usuários}
\begin{description}
\item[Escopo:] Sistema Back-end
\item[Nível:] Meta do Administrador
\item[Ator Primário:] Administrador
\item[Stakeholders e seus interesses:] \hfill \\
Administrador: Deseja remover administradores ou atendentes.

\item[Pré-condições:] O sistema está operante. O Administrador está logado.
\item[Pós-condições:] O usuário foi removido.
\item[Fluxo Básico ou Cenário Principal:]\hfill
\begin{enumerate}
  \item O Administador fornece o login do usuário a ser removido.
  \item A operação é executada.
\end{enumerate}

\item[Extensões:]\hfill
\begin{description}
	\item[1a.] O usuário não existe.
	\begin{enumerate}
 		\item O Back-end informa ao Administrador que o usuário não existe.
	\end{enumerate}

\end{description}
\item[Requisitos especiais:] Nenhum.
\item[Tecnologia:] Nenhuma.
\item[Freqüência de Ocorrência:] Raramente.

\end{description}

\section{Back-end: Alterar Usuários}
\begin{description}
\item[Escopo:] Sistema Back-end
\item[Nível:] Meta do Administrador
\item[Ator Primário:] Administrador
\item[Stakeholders e seus interesses:] \hfill \\
Administrador: Deseja alterar administradores ou atendentes.

\item[Pré-condições:] O sistema está operante. O Administrador está logado.
\item[Pós-condições:] O usuário foi alterado.
\item[Fluxo Básico ou Cenário Principal:]\hfill
\begin{enumerate}
  \item O Administador fornece o login do usuário a ser alterado.
  \item O Administrador altera os campos de login, senha ou grupo do
  usuário.
  \item A operação é executada.
\end{enumerate}

\item[Extensões:]\hfill
\begin{description}
	\item[1a.] O usuário não existe.
	\begin{enumerate}
 		\item O Back-end informa ao Administrador que o usuário não existe.
	\end{enumerate}
	\item[2a.] O login e/ou senha estão em branco.
	\begin{enumerate}
 		\item O Back-end informa ao Administrador que o campo está em branco.
 		\item O Back-end impede a operação de ser concluída enquanto o campo
 		permancer em branco.
	\end{enumerate}	

\end{description}
\item[Requisitos especiais:] Nenhum.
\item[Tecnologia:] Nenhuma.
\item[Freqüência de Ocorrência:] Raramente.

\end{description}



\chapter{Requisitos Não Funcionais}
\begin{itemize}
  \item O sistema deve ser multiplataforma
  \item O sistema não possui restrição de desempenho maior que 1 segundo de
  tempo de resposta
  \item O Front-end deve ser o mais simples e intuitivo quanto possível
%  \item  Os Back-ends devem possuir um sistema de controle de acesso que evite
%  usos indevidos do sistema.
\end{itemize}

\chapter{Glossário}
\begin{itemize}
  \item LabUFSC - Laboratório de Informática da UFSC
  \item Front-end: Parte do sistema visível ao Usuário
  \item Back-end: Parte do sistema visível ao Administrador
  \item Back-end Secretaria: Parte do sistema visível ao Atendente.
  \item Usuário: Usuário comum das máquinas do laboratório.
  \item Administrador: Membro da equipe de manutenção do laboratório.
  \item Atendente: Membro da equipe da secretaria do laboratório.
\end{itemize}

\chapter{Especificação Suplementar}
\section{Regras para verificação de matrícula}
\begin{itemize}
  \item Deve ser constituída apenas de números
  \item Tamanho de 6 ou 8 caracteres
\end{itemize}

%\bibliographystyle{abnt-alf}
%\bibliography{bibliografia}
\end{document}