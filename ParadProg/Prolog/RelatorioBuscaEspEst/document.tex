\documentclass[brazil,times]{abnt}
\usepackage[T1]{fontenc}
\usepackage[utf8]{inputenc}
\usepackage{url}
\usepackage{graphicx}
\usepackage[pdfborder={0 0 0}]{hyperref}
\usepackage{listingsutf8}
\makeatletter
\usepackage{babel}
\makeatother

\lstset{
language=Prolog,
tabsize=2,
inputencoding=utf8,
basicstyle=\scriptsize,
showspaces=false,
showstringspaces=false,
showtabs=false,
% columns=fullflexible
}

\begin{document}

\autor{Pedro Paulo Vezzá Campos \\ Rafael Elias Pedretti}

\titulo{Implementação de Problema de Busca em Espaço de Estados utilizando
Prolog}

\comentario{Trabalho apresentado para avaliação na disciplina INE5416, do
curso de Bacharelado em Ciências da Computação, turma 04208, da Universidade   
Federal de Santa Catarina, ministrada pelos professores João Cândido Lima
Dovicchi e Jerusa Marchi}

\instituicao{Departamento de Informática e Estatística \par Centro
Tecnológico \par Universidade Federal de Santa Catarina}

\local{Florianópolis - SC, Brasil}

\data{\today}

\capa

\folhaderosto

% \tableofcontents
%\chapter{}
\section*{Introdução}
	Para este trabalho final de INE5416 foi proposto pela professora a
	escolha e implementação em Prolog de um problema de busca em espaço de estados.
	Os alunos escolheram o problema do lobo, da ovelha e da alface, um problema
	clássico de travessia de rio, que remonta pelo menos ao século IX
	\cite{pressman:rivercrossing}.
	
	Este relatório está organizado da seguinte forma: Primeiramente será
	apresentada uma descrição do problema a ser modelado e resolvido. Depois será
	apresentada a definição dos estados inicial, final e operadores envolvidos. Em
	seguida será exiblida a representação dos estados em Prolog. Posteriormente
	será visualizada a implementação da função geradora de sucessores. Por fim, os
	alunos apresentarão algumas considerações sobre a execução do projeto.

\section*{Descrição do Problema Escolhido}
	Um fazendeiro viajava com três compras: Um lobo, uma ovelha e alface. Ao
	retornar para casa ele se deparou com uma margem de um rio, arrendando um barco
	para realizar a travessia. Porém, o barco arrendado era pequeno e somente
	comportava o fazendeiro e apenas uma de suas compras.
	
	Enquanto viajavam junto do fazendeiro nenhuma compra atacava a outra, no
	entanto, assim que fossem deixados sozinhos o lobo comeria a ovelha e a
	ovelha comeria a alface.
	
	O problema consiste em terminar a travessia tanto do fazendeiro
	quanto de suas compras para a margem oposta sem permitir que uma compra ataque
	a outra.

\section*{Definição dos Estados Inicial, Final e Operadores}
	


\section*{Representação dos Estados}


\section*{Implementação da Função Geradora de Sucessores}


\section*{Considerações sobre o Trabalho}


\chapter*{Código Fonte}
	\lstinputlisting{src/PedroPaulo_RafaelPedretti_TF.pl}

\bibliographystyle{abnt-num}
\bibliography{bibliografia}
\end{document}