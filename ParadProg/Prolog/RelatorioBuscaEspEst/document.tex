\documentclass[brazil,times]{abnt}
\usepackage[T1]{fontenc}
\usepackage[utf8]{inputenc}
\usepackage{url}
\usepackage{graphicx}
\usepackage[pdfborder={0 0 0}]{hyperref}
\usepackage{listingsutf8}
\usepackage{amssymb}
\makeatletter
\usepackage{babel}
\makeatother

\lstset{
language=Prolog,
tabsize=2,
inputencoding=utf8,
basicstyle=\scriptsize,
showspaces=false,
showstringspaces=false,
showtabs=false,
% columns=fullflexible
}

\begin{document}

\autor{Pedro Paulo Vezzá Campos \\ Rafael Elias Pedretti}

\titulo{Implementação de Problema de Busca em Espaço de Estados Utilizando
Prolog}

\comentario{Trabalho apresentado para avaliação na disciplina INE5416, do
curso de Bacharelado em Ciências da Computação, turma 04208, da Universidade   
Federal de Santa Catarina, ministrada pelos professores João Cândido Lima
Dovicchi e Jerusa Marchi}

\instituicao{Departamento de Informática e Estatística \par Centro
Tecnológico \par Universidade Federal de Santa Catarina}

\local{Florianópolis - SC, Brasil}

\data{\today}

\capa

\folhaderosto

% \tableofcontents
%\chapter{}
\section*{Introdução}
	Para este trabalho final de INE5416 foi proposto pela professora a
	escolha e implementação em Prolog de um problema de busca em espaço de estados.
	Os alunos escolheram o problema do lobo, da ovelha e da alface, um problema
	clássico de travessia de rio, que remonta pelo menos ao século IX
	\cite{pressman:rivercrossing}.
	
	Este relatório está organizado da seguinte forma: Primeiramente será
	apresentada uma descrição do problema a ser modelado e resolvido. Depois será
	apresentada a definição dos estados inicial, final e operadores envolvidos. Em
	seguida será exiblida a representação dos estados em Prolog. Posteriormente
	será visualizada a implementação da função geradora de sucessores. Ainda, os
	alunos apresentarão algumas considerações sobre a execução do projeto. Por fim
	será apresentado o código fonte completo e documentado da aplicação finalizada.

\section*{Descrição do Problema Escolhido}
	Um fazendeiro viajava com três compras: Um lobo, uma ovelha e alface. Ao
	retornar para casa ele se deparou com uma margem de um rio, arrendando um barco
	para realizar a travessia. Porém, o barco arrendado era pequeno e somente
	comportava o fazendeiro e no máximo uma de suas compras.
	
	Enquanto viajavam junto do fazendeiro nenhuma compra atacava a outra, no
	entanto, assim que fossem deixados sozinhos o lobo comeria a ovelha e a
	ovelha comeria a alface.
	
	O problema consiste em terminar a travessia tanto do fazendeiro
	quanto de suas compras para a margem oposta sem permitir que uma compra ataque
	a outra. \cite{ptwiki:problema-lobo-ovelha-alface}

\section*{Definição dos Estados Inicial, Final e Operadores}
	Conforme explicado anteriormente o problema consiste em mover todos os itens de
	uma margem a outra, portanto foram modelados ambos os lados do rio da seguinte
	forma:
	
	\begin{description}
		\item[Estado Inicial] \hfill \\
		Margem direita: \{barco, lobo, ovelha, alface\} \\
		Margem esquerda: $\varnothing$
		\item[Estado Final] \hfill \\
		Margem direita: $\varnothing$ \\
		Margem esquerda: \{barco, lobo, ovelha, alface\}
		\item[Operadores] \hfill
			\begin{description}
				\item[\texttt{moveOvelha/2}] Este predicado a sempre irá realizar a
				mudança da ovelha e do barco de margem, uma vez que o problema não
				impõe restrições ao lobo estar junto da alface.
				\item[\texttt{moveLobo/2}] Este predicado realiza a mudança do lobo e do
				barco de uma margem a outra somente se a ovelha e a alface não ficarem
				sozinhas em um lado do rio
				\item[\texttt{moveAlface/2}] Este predicado realiza a mudança da alface e do
				barco de uma margem a outra somente se o lobo e a ovelha não ficarem
				sozinhos em um lado do rio 
				\item[\texttt{moveBarco/2}] Este predicado realiza a mudança somente do
				barco de uma margem a outra com a condição que tanto o lobo e a ovelha
				quanto a ovelha e a alface não ficarão sozinhos em um lado do rio 
			\end{description}
	\end{description}

\section*{Representação dos Estados}
	Para a representação em Prolog foram adotadas listas de átomos como
	estrutura de dados principal devido a sua flexibilidade e a presença de uma
	biblioteca de predicados para sua manipulação produzida como atividade
	prática da disciplina durante o semestre. Os estados do problema estão assim
	representados:
	
	\begin{description}
		\item[Estado Inicial] \texttt{[[],[barco,ovelha, lobo, alface]]}
		\item[Estado Intermediário] Alguma combinação dos quatro itens distribuídos
		nas duas listas segundo as condições abaixo, tal como \texttt{[[lobo,
		alface],[barco, ovelha]]}
		\item[Estado Final] \texttt{[[barco,ovelha, lobo, alface],[]]}
	\end{description}

	Por definição, há uma lista mais externa que comporta duas listas internas. A
	primeira lista interna contém os itens presentes na margem direita e a segunda
	os itens da margem esquerda. Tais listas são mutuamente exclusivas, sendo a
	união de ambas sempre igual a \texttt{[barco,ovelha, lobo, alface]}.
	
	%TODO: A ordem dos elementos nas listas importa?
	
\section*{Implementação da Função Geradora de Sucessores}
	

\section*{Considerações sobre o Trabalho}
	Este trabalho final de INE5416 proposto pela professora Jerusa foi de grande
	valia por apresentar como proposta a escolha de um exercício dentre um rol de
	problemas clássicos que permitem evidenciar ao aluno as vantagens trazidas pelo
	uso do paradigma lógico para a solução de diversos problemas.

	Através deste trabalho foi possível experimentar diversos problemas inerentes à
	programação em um paradigma completamente diferente do habitual, treinando as
	habilidades na abordagem e solução dessas dificuldades.
	
	Ainda, este trabalho contribuiu ao resumir bem diversos conceitos vistos
	durante o semestre abrangendo desde recursão, unificação e listas até culminar no
	projeto, desenvolvimento e testes de uma aplicação prática. Por outro lado,
	como os exercícios não demandaram grandes e tediosos esforços de programação os
	acadêmicos puderam concentrar-se nas modelagens e abordagens de solução
	adotadas para resolver o problema, as quais foram apresentadas anteriormente
	neste relatório.


\chapter*{Código Fonte}
	\lstinputlisting{src/PedroPaulo_RafaelPedretti_TF.pl}

\bibliographystyle{abnt-num}
\bibliography{bibliografia}
\end{document}